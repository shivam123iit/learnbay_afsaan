\documentclass[aspectratio=169, 10pt]{beamer}
\usepackage{graphicx}
\usepackage{lingmacros}
\usepackage{tree-dvips}
\usetheme{Madrid}
\usepackage{multirow}
\usepackage{multicol}

\begin{document}


\begin{frame}{Git and Git Hub}{Afsaan,Learbay}

1)Git and Git hub comes under VCS-Version Control System.\\
2)VS Code is 98 percent Company for Development.\\
3)Jupyter Notebook is starting.\\
4)BitBucket.\\
5)Google Collab.\\

\vspace{1 cm}
VS Code is for any coding.\\
\end{frame}

\begin{frame}[t]{Git and Git Hub}{Afsaan,Learbay}
Command Line basics:\\  \vspace{0.25 cm}
1) ls(List) is command used for list in that directory.\\
2) ls -a (List) list with hidden files.
3) pwd(Present working directory) is present working directory.\\
4) cd change Directory is present working directory.\\
5)Ctrl+C for out of loop\\
6)Tab for help in selecting  file or folder\\
7)mkdir for make directory.\\
8)rmdir for removing directory, it should be empty.\\
9)rm -r for deleting any directory irrespective of empty or some file.\\
\hspace*{1cm} rm -r path file.extension\\
10)Create file by touch command, ex- touch shivam.txt  
  
if folder name has space then syntax is cd 'Shivam Asthana'

\end{frame}

\begin{frame}[t]{Git and Git Hub}{Afsaan,Learbay}
Git basics:\\  \vspace{0.25 cm}
git config -\hspace{1mm}-global user.name "Shivam123iit"\\
git config -\hspace{1mm}-global user.email "shivam.asthana123@gmail.com"\\
git config -\hspace{1mm}-global -\hspace{1mm}-list\\
git clone    https://github.com/shivam123iit/learnbay\_afsaan.git\\
git config -\hspace{1mm}-global https.sslverify false\\
git status \\
git add \\
git commit My file.pdf -m 'My first commit' \\ 
git push origin master \\
git add . Adds all files to stagging area\\
git log -- To get Log of all commits \\
git diff -- Will tell about diff in files\\
 
\end{frame}
\begin{frame}[t]{Python Components}{Afsaan, Learnbay}
1)\hspace{1 cm}Literals\\
2)\hspace{1 cm}Constants\\
3)\hspace{1 cm}Variables\\
4)\hspace{1 cm}Identifier\\
5)\hspace{1 cm}Reserve Words\\
6)\hspace{1 cm}Statement and Expression\\
7)\hspace{1 cm}Block and Identation\\
8)\hspace{1 cm}Comments. \\
\end{frame}

\begin{frame}[t]{LITERALS}
\textbf{Literals :Value that we use in Python is a literals, eg Integer, float, etc.}\\
\vspace{0.5 cm}
Integer 10, 2, 9999 etc.\\
float 10.2, 2.3, 4.35 etc.\\
Complex number : Anything which can be written in form of a+bj.\\
None \\
Boolean : True and False.\\
String : Any thing in single Quote '....', Double Quote "...." and triple Single Quote '''.....''' and triple double quotes """...."""\\
\vspace{0.5 cm}
\textbf{Examples:}\\
a=10, a is a variable, = is a operator and 10 is a Literal.
b= True , this is boolean.\\
c= None , this is None\\
d= 'Shivam' , this is a String Literal.\\ 
\end{frame}

\begin{frame}[t]{Identifiers}
\textbf{Identifiers: Any Name which we give in Python is a Identifier , this can be variable name, Constant Name, function Name and Class Name}\\
\vspace{0.5 cm}
Rules:\\
1) Anything A-Z, a-z, 0-9 and \_\\
2) Special Characters are not allowed.\\
3) It should not start with Number.\\ 
\end{frame}

\end{document}